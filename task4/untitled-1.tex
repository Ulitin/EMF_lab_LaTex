\documentclass[12pt,a4paper]{article}
\usepackage[utf8]{inputenc}
\usepackage[english,russian]{babel}
\usepackage{indentfirst}
\usepackage{misccorr}
\usepackage{graphicx}
\usepackage{amsmath}
\usepackage{latexsym}
\usepackage{cmap}
\usepackage{graphicx}



\begin{document}
\section{Теоретическое задание 4}
Решение $u$ называется стационарным, если оно не зависит от времени $u(x, t) = u(x)$ (предпологаем, что $f(x, t) \equiv f(x)$). Для стационарного решения, очевидно, $\frac{du}{dt} \equiv 0$.

\begin{equation*}
    a^2 u_{xx} -f(x) = 0 \Rightarrow
    \eqno(1)\
\end{equation*}
\begin{equation*}
    \Rightarrow \delta u = -f(x) -
\end{equation*}
- уравнение Пуассона (элептическое д.у.).
Если $f(x) \equiv 0$, то (1) - уравнение Лаппласа.\\
1) Начальные условия на установление стационарного режима не влияют.\\
2) Граничные условия:\\
    \par $\sim \frac{1}{t}$ - устанавливается стационарное равновесие\\
    \par $\sim cos(), sin()$ - не устанавливается\\
3) Если $f(x, t)$ зависит только от x или является константой, то со временем к стационарному режиму придет.\\
В противном случаи, сложно ожидать, что стационарный режим установится.

\begin{center}
\begin{equation*}
    u_{o.o.} = C_1 x + C_2
\\
    a^2 u_{xx} - f(x) = 0 \Leftrightarrow u_{xx} = \frac{f(x)}{a^2}
\\
    u_{o.n.} = C_1 x + C_2 - \frac
        {\int\limits_0^x {\int\limits_0^y f(\xi) \, d\xi} \, dy}
        {a^2}
\end{equation*}
\end{center}

Граничные условия:
\begin{center}
    $\left[
        \begin{gathered}
                u(0) = \mu_1(x) \\
                u(l) = \mu_2(x) \\
        \end{gathered}
    \right.$
\end{center}

\begin{center}
    \begin{gathered}
        u(0) = \mu_1(x)
\\
        \mu_1(x) = C_2
\\
        \mu_2(x) = C_1 l + C_2 - \frac
            {\int\limits_0^l {\int\limits_0^y f(\xi) \, d\xi} \, dy}
            {a^2} 
            = C_1 l + \mu_1(x) - \frac
            {\int\limits_0^l {\int\limits_0^y f(\xi) \, d\xi} \, dy}
            {a^2}
\\
            C_1 = \frac{-\mu_1(x)}{l} - \frac
            {\int\limits_0^l {\int\limits_0^y f(\xi) \, d\xi} \, dy}
            {a^2 l}
            + \frac{\mu_2(x)}{l}
\\
    \end{gathered}
\end{center}

\begin{center}
\begin{equation*}
        M_{o.n.} = \frac{1}{l} 
        \left(
            \frac{-\mu_1(x)}{l} - 
            \frac
                {\int\limits_0^l {\int\limits_0^y f(\xi) \, d\xi} \, dy}
                {a^2}
            + \mu_2(x)
        \right)
        x + \mu_1(x) -
        \frac
            {\int\limits_0^l {\int\limits_0^y f(\xi) \, d\xi} \, dy}
            {a^2}
\end{equation*}
\end{center}

Проверка:

\begin{center}
\begin{gathered}
    M(0) = \mu_1 \\
    M(l) = -\mu_1(x) +
    \frac
        {\int\limits_0^l {\int\limits_0^y f(\xi) \, d\xi} \, dy}
        {a^2}
    + \mu_2 - \mu_1
    \frac
        {\int\limits_0^l {\int\limits_0^y f(\xi) \, d\xi} \, dy}
        {a^2}
    = \mu_2
\end{gathered}
\end{center}

\end{document}