\documentclass[12pt,a4paper]{article}
\usepackage[utf8]{inputenc}
\usepackage[english,russian]{babel}
\usepackage{indentfirst}
\usepackage{misccorr}
\usepackage{graphicx}
\usepackage{amsmath}
\usepackage{latexsym}
\usepackage{cmap}


\begin{document}
\section{Вывод уравнения теплопроводности}
Рассмотрим бесконечно малый кусок стержня длинной $\Delta x$ с осью $OX$ вдоль стержня. На левом конце стержня подается тепло. Количество входимой теплоты на промежуток $\Delta t$ назовем $Q_{in}$. Тепло будет расспространяться по всему стержню. В результате теплообмена, на другом конце стержня будет выделяться тепло. Назовем выделенное тепло за промежуток $\Delta t$ на другом конце стержня  $Q_{out}$. Введем пременную

\begin{displaymath}
\Delta Q = Q_{in} - Q_{out}
\eqno(1)\
\end{displaymath}

$\Delta Q$ - это потеря тепла на нагрев участка $\Delta x$. Согласно закону Фурье количество тепла, протекающего в направлении оси $OX$ за бесконечно малый промежуток времени $dt$ через сечение $S$ с абсциссой $x$, будет:

\begin{displaymath}
dQ = -k S \frac{\partial u(x,t)}{\partial x} \Delta t
\eqno(2)\
\end{displaymath}

, где $k$ – коэффициент теплопроводности, $S$ - площадь поперечного чесения стержня, а функция $u$ показывает температуру стержня в точке $x$ во время $t$  ( $\frac{\partial u(x,t)}{\partial x}$ представляет здесь величину градиента температуры U). В этой формуле стоит знак минус, так как при $\frac{\partial u(x,t)}{\partial x} > 0$, т.е. при росте $u$ вместе с $x$, поток тепла направлен в противоположную сторону). Воспользуемся формулой (2) ирасспишем  $Q_{in}$ и  $Q_{out}$.

\begin{displaymath}
dQ_{in} = -k S \frac{\partial u(x,t)}{\partial x} \bigg|_x \Delta t
\eqno(3)\
\end{displaymath}

\begin{displaymath}
dQ_{out} = -k S \frac{\partial u(x,t)}{\partial x} \bigg|_{x+ \Delta x} \Delta t
\eqno(4)\
\end{displaymath}

Подставим формулы (4) и (3) в (1) и получим: 

\begin{displaymath}
\Delta Q = \Delta t k S * [ \frac{\partial u(x,t)}{\partial x} \bigg|_{x+ \Delta x} - \frac{\partial u(x,t)}{\partial x} \bigg|_x]
\eqno(5)\
\end{displaymath}

Напишем формулу для нагрева стержня длинной $\Delta x$, теплоемкостью $c$ и массой $m$.

\begin{displaymath}
\Delta Q = с m \frac{\partial u(x,t)}{\partial t} \Delta t
\eqno(6)\
\end{displaymath}

Т.к. $m = \rho V$, а $V = S \Delta x$, то формула (5) перепишется в таком виде:

\begin{displaymath}
\Delta Q = с \rho S \Delta x \frac{\partial u(x,t)}{\partial t} \Delta t
\eqno(7)\
\end{displaymath}

Приравняем формулы (5) и (7):

\begin{displaymath}
с \rho S \Delta x \frac{\partial u(x,t)}{\partial t} \Delta t =  \Delta t k S * [ \frac{\partial u(x,t)}{\partial x} \bigg|_{x+ \Delta x} - \frac{\partial u(x,t)}{\partial x} \bigg|_x]
\eqno(8)\
\end{displaymath}

Поделим равенство на $с \rho S \Delta x \Delta t$. Справа получим производную 

\begin{displaymath}
\lim_{\Delta x \to 0} \frac{\frac{\partial u(x,t)}{\partial x} \bigg|_{x+ \Delta x} - \frac{\partial u(x,t)}{\partial x} \bigg|_x}{\Delta x}= \frac{\partial^2 u(x,t)}{\partial x^2}
\end{displaymath}

Сделем замену $a^2 = k/\rho$. Формула (8) перепишеться в другом виде:

\begin{displaymath}
\frac{\partial u(x,t)}{\partial t} =  a^2  \frac{\partial^2 u(x,t)}{\partial x^2}
\eqno(9)\
\end{displaymath}

\end{document}